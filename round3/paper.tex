\documentclass[amsmath,amssymb,twocolumn,superscriptaddress,aps,prl]{revtex4-1}

\usepackage{lipsum}
\usepackage{graphicx}

\begin{document}

\title{Why you should be writing a reproducible paper}

\author{A.~R.~McCluskey}
\email{a.r.mccluskey@bath.ac.uk}
\affiliation{Department of Chemistry, University of Bath, Claverton Down, Bath, BA2 7AY, UK}
\affiliation{Diamond Light Source, Harwell Campus, Didcot, OX11 0DE, UK}

\date{\today}

\begin{abstract}
  Following many years of irreproducible research it is time to make a change, this paper will convince you why.
\end{abstract}

\maketitle

What is the point in science if we cannot share it?
What is the point in sharing if we cannot reproduce it?
Tania Allard has a great talk about reproducible science with Python \cite{Allard2018}.

\begin{figure}
  \centering
  \includegraphics[width=0.45\textwidth]{figure1.pdf}
  \caption{\small This is the raw data.}
  \label{fig:raw}
\end{figure}

\lipsum

\bibliography{paper2}

\end{document}
